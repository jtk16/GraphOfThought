\documentclass[11pt,onecolumn,letterpaper]{article}
\usepackage[utf8]{inputenc}
\usepackage{amsmath,amssymb,amsfonts}
\usepackage{mathtools}
\usepackage{graphicx}
\usepackage{enumitem}
\usepackage{algorithm}
\usepackage{algpseudocode}
\usepackage{booktabs}
\usepackage{hyperref}
\usepackage{microtype}
\usepackage{caption}
\usepackage{subcaption}
\usepackage{tikz}
\usepackage{listings}
\usepackage{cleveref}

\title{\Large \bf KV-Cache Thought Graphs:\\
A Scalable, Parallel, and Auditable Architecture for Non-Linear LLM Reasoning}

\author{Author Name \\
Institution / Team \\
\texttt{email@example.com}
}
\date{\today}

\begin{document}
\maketitle

\begin{abstract}
Large language models (LLMs) often rely on linear chain-of-thought (CoT) traces or implicit hidden states to perform multi-step reasoning. Linear CoT limits parallelism, revisitation, and explicit auditing. We introduce the \emph{KV-Cache Thought Graph} (KVTG), a principled architecture that represents an LLM's internal reasoning as a versioned directed graph of immutable \emph{thought nodes}, each associated with a compact KV-cache snapshot (or compressed proxy). Nodes are expanded by worker controllers, and the system supports parallel speculative expansion, differential copy-on-write storage, probabilistic expansion policies, and multiple merge strategies (intersection, union, weighted voting, adversarial critique). KVTG permits safe parallelism (no shared mutable KV), fine-grained memory management (active/recent/cold tiers with adaptive compression), and systematic verification via synthesizers and verifiers. We provide formal specifications, algorithms, system designs, and a prototype experimental roadmap. We analyze complexity, failure modes, and trade-offs, and we propose evaluation tasks (math proofs, code generation, hypothesis testing) to quantify the benefits of graph reasoning versus linear CoT and tree-of-thoughts baselines.
\end{abstract}

\section{Introduction}
Chain-of-thought (CoT) prompting and internal hidden activations have enabled LLMs to solve complex, multi-step tasks. Standard CoT is linear: intermediate steps are written sequentially, and the model proceeds token-by-token. This linearity imposes key disadvantages:
\begin{itemize}[nosep]
  \item \textbf{Limited parallelism:} reasoning branches cannot be expanded concurrently without duplicating context.
  \item \textbf{Poor revisitation:} returning to an earlier intermediate step requires replaying context or re-computing hidden state.
  \item \textbf{Opaque internals:} the hidden state is a transient monologue; auditing, provenance, and modular verification are limited.
\end{itemize}

We propose \emph{KV-Cache Thought Graphs} (KVTG): represent reasoning as a directed graph \(G=(V,E)\) where each node \(v\in V\) is a \emph{thought node} containing an immutable snapshot of the LLM's KV-cache state (or a compressed summary/proxy), a textual summary, and metadata. Edges encode semantic, causal, or evidential relationships. This design enables: (1) safe parallel expansion of nodes (workers operate on immutable snapshots or deltas); (2) revisitation by rehydrating nodes into the active context; (3) richer merges and verification protocols; (4) flexible memory management via differential copy-on-write (CoW) and multi-tier storage.

This paper presents a full specification of KVTG: data structures, controller architecture, probabilistic expansion policies, merge strategies, storage/compression recipes, and experimental plans. We analyze performance trade-offs, provide pseudocode and an API sketch, and discuss implementation pitfalls and mitigations.

\section{Related Work}
(Condensed — expand as needed for publication.) KVTG touches on:
\begin{itemize}[nosep]
  \item \textbf{Chain-of-Thought (CoT)} prompting and Tree-of-Thoughts (ToT): prior works use token-level or branch-level search to explore reasoning paths. KVTG generalizes ToT by storing explicit KV snapshots and designing system-level storage/merge mechanisms.
  \item \textbf{KV-cache and sparse attention systems:} vLLM, FlashAttention, and paging architectures motivate paged KV designs and coalesced attention.
  \item \textbf{Neural memory / episodic memory:} indexable memories with different precision tiers inform our storage hierarchy and compression strategies.
  \item \textbf{Parallel search / MCTS:} KVTG's probabilistic expansion and voting merges parallelize exploration but emphasize immutable snapshots and verifiable merges.
\end{itemize}

\section{Modeling Thought Graphs}
\subsection{Thought nodes and metadata}
A thought node \(v\) is a tuple:
\[
v = \big(\text{text}_v,\, \text{sum}_v,\, \text{emb}_v,\, \text{kv\_ptr}_v,\, \text{meta}_v\big)
\]
\begin{itemize}[nosep]
  \item \(\text{text}_v\): raw text content (the tokens generated/observed).
  \item \(\text{sum}_v\): short human- or model-readable summary.
  \item \(\text{emb}_v\in\mathbb{R}^d\): semantic embedding for retrieval, clustering.
  \item \(\text{kv\_ptr}_v\): pointer(s) to KV-cache pages or compressed proxy.
  \item \(\text{meta}_v\): structured metadata (timestamp, creator worker, provenance, confidence).
\end{itemize}

Nodes are \emph{immutable} after creation. New reasoning from node \(v\) produces a child node \(v'\) linking to \(v\) (or multiple parents), optionally storing only KV deltas for efficiency.

\subsection{Edges}
Edges \(e=(v_i,v_j)\) carry:
\[
e = \big(\text{rel\_type},\, \text{confidence},\, \text{provenance}\big)
\]
where \(\text{rel\_type}\) may be among: \(\{\text{supports}, \text{weakly\_supports}, \text{contradicts}, \text{causal\_before}, \text{correlational}, \ldots\}\).

\section{Storage, Compression, and Copy-on-Write}
\subsection{Three-tier storage}
We adopt a tiered storage model per node:
\begin{enumerate}[nosep]
  \item \textbf{Active (VRAM)}: full precision KV pages for nodes currently being expanded.
  \item \textbf{Recent (DRAM)}: reduced-precision/quantized KV proxies and higher-fidelity text+embeddings.
  \item \textbf{Cold (SSD/DB)}: highly compressed summaries, embeddings, and tiny proxies for recall.
\end{enumerate}

\subsection{Differential KV and CoW}
Let \(\text{KV}(v)\) denote the set of KV pages for node \(v\). When creating child node \(v'\) from \(v\), we store:
\[
\text{KV}(v') = \text{KV}(v) + \Delta(v\to v').
\]
Only pages modified are materialized; unmodified pages are referenced (CoW). Deltas are stored sparsely (page-level) to reduce duplication. If a worker writes to a shared page, a copy is created (atomic CoW). This enables cheap branching and efficient memory usage for small edits.

\subsection{Adaptive compression}
For a node \(v\), compute an attention-profile vector \(\alpha_v\) indicating which KV pages were most used when creating \(v\). Use this to select compression:
\begin{itemize}[nosep]
  \item Keep top-k pages in higher precision.
  \item Quantize lower-importance pages (e.g., 8-bit, 4-bit) or approximate them with low-rank factors.
  \item Build a compact summary proxy (summary text + small embedding) for cold storage.
\end{itemize}

\section{Controller Architecture}
A major design choice is how to schedule node expansions. We propose a \emph{hybrid distributed controller}:

\subsection{Workers and local controllers}
Each worker (GPU process) manages:
\begin{itemize}[nosep]
  \item A local queue of nodes assigned to it.
  \item Full control for expansion decisions inside its partition (local policy).
  \item Asynchronous reporting of candidate expansions and summaries to the meta-controller.
\end{itemize}

\subsection{Meta-controller and consensus}
Meta-controller aggregates proposals and decides cross-partition resource allocation and global merges. Instead of a single hard lock-based controller, we use a \emph{soft consensus}:
\begin{itemize}[nosep]
  \item Workers propose top-\(k\) candidates \(C_i=\{v_{i,1},\ldots,v_{i,k}\}\) with scores.
  \item Meta-controller computes a weighted soft-vote and returns accepted expansions.
  \item The meta-controller provides budgeted expansion slots; workers then fetch KV snapshots and expand.
\end{itemize}

This hybrid design reduces a single point of contention while enabling global coordination for merges and memory management.

\section{Probabilistic Expansion Policy}
We formalize a probabilistic selection rule that accounts for uncertainty, utility, recency, and cost.

For node \(i\) define:
\begin{itemize}[nosep]
  \item \(u_i\): uncertainty estimate (e.g., disagreement among multiple proposals).
  \item \(v_i\): expected utility (learned evaluator or heuristic).
  \item \(p_i\): recency or priority decay factor.
  \item \(c_i\): estimated expansion cost (I/O / prefetch latency).
\end{itemize}

Define the expansion probability:
\begin{equation}\label{eq:prob}
P(\text{expand } i) = \frac{\exp(\alpha u_i + \beta v_i + \gamma p_i - \delta c_i)}{\sum_j \exp(\alpha u_j + \beta v_j + \gamma p_j - \delta c_j)}.
\end{equation}

Hyperparameters \(\alpha,\beta,\gamma,\delta\) can be tuned (search or RL). This policy balances exploration (high \(u_i\)) against exploitation (high \(v_i\)) and system constraints (penalize high \(c_i\)).

\section{Merge Strategies}
When branches become relevant to each other (e.g., they address a common question or their nodes are mutually reachable), we offer multiple merge strategies:
\begin{itemize}[nosep]
  \item \textbf{Intersection:} conservative, keep common validated content.
  \item \textbf{Union + Summarize:} combine unique insights then compress.
  \item \textbf{Weighted Voting:} treat each branch as an expert; accept conclusions above threshold.
  \item \textbf{Adversarial Merge:} spawn critic tasks to find contradictions or counterexamples; accept only robust conclusions.
\end{itemize}

Merges are performed by \emph{synthesizer} LLMs (which read branch summaries) plus \emph{verifiers} (symbolic or learned checkers). Only verified items are allowed to become high-fidelity nodes; others remain as low-fidelity hypotheses.

\section{Algorithms and Pseudocode}
We provide two core algorithms: (A) node expansion cycle on a worker, and (B) meta-controller soft-vote aggregation.

\begin{algorithm}[H]
\caption{Worker Expansion Loop}
\label{alg:worker}
\begin{algorithmic}[1]
\State \textbf{Input:} local queue \(Q\), local KV storage
\While{budget not exhausted}
  \State Fetch node \(v\) from \(Q\) or receive assignment
  \State Ensure \(\text{KV}(v)\) are resident (prefetch if needed)
  \State Execute LLM expansion on \(\text{KV}(v)\) to produce children \(\{v'\}\)
  \For{each child \(v'\)}
    \State Compute \(\text{emb}_{v'}\), \(\text{sum}_{v'}\), \(u_{v'}\), \(v_{v'}\), \(c_{v'}\)
    \State Store node metadata, and persist KV deltas in Active tier (CoW)
    \State Report top candidates to meta-controller
  \EndFor
  \State Optionally synthesize/verify and mark nodes as Dormant/Active
\EndWhile
\end{algorithmic}
\end{algorithm}

\begin{algorithm}[H]
\caption{Meta-Controller Soft Vote Aggregation}
\label{alg:meta}
\begin{algorithmic}[1]
\State \textbf{Input:} proposals \(\{C_i\}\) from workers (each with scores)
\State Aggregate proposals into candidate set \(C=\bigcup_i C_i\)
\For{each candidate \(c\in C\)}
  \State Compute aggregated score \(s_c\) (weighted by worker reliability, novelty)
\EndFor
\State Select top-\(M\) candidates by soft-vote for assignment
\State Issue assignments to respective workers and prefetch directives
\end{algorithmic}
\end{algorithm}

\section{Prototype and Experimental Roadmap}
We recommend the following staged evaluation to validate KVTG benefits.

\subsection{Prototype 1: Thought Forest (single-machine)}
\begin{itemize}[nosep]
  \item \textbf{LLM:} mid-size open model (e.g., LLaMA-7B) served by a paged KV engine or simulated KV proxies.
  \item \textbf{Storage:} implement CoW delta pages in main memory; emulate VRAM/DRAM/SSD tiers by controlling precision and simulated access latencies.
  \item \textbf{Controller:} implement local workers + simple meta soft-vote aggregator.
  \item \textbf{Tasks:} arithmetic + algebra proofs (multi-step), code generation (alternate implementations), short scientific hypothesis generation.
  \item \textbf{Metrics:} wall-clock throughput (proofs/hour), memory usage, number of node expansions, success rates, reproducibility/interpretability (qualitative).
\end{itemize}

\subsection{Prototype 2: Prefetching and Compression}
Add adaptive compression and speculative prefetching; measure stall time reductions and trade-offs between compression and correctness.

\subsection{Prototype 3: Distributed Partitioning}
Partition graph across machines; evaluate cross-partition merge latency and global scaling.

\section{Evaluation Metrics}
We measure:
\begin{itemize}[nosep]
  \item \textbf{Task performance:} solution accuracy (proof correctness, code test pass rate).
  \item \textbf{Efficiency:} wall-clock time to solution, GPU utilization, memory usage, average expand latency.
  \item \textbf{Parallelism:} speedup vs linear baseline (CoT) and ToT.
  \item \textbf{Robustness:} how merging strategies affect correctness and overconfidence.
  \item \textbf{Auditability:} human interpretability scores for produced thought graphs; ease of tracing provenance.
\end{itemize}

\section{Complexity and Resource Analysis}
Let \(B\) be batch/worker count, \(n\) KV pages per node, \(r\) average delta pages per child, and \(q\) active nodes. Costs:
\begin{itemize}[nosep]
  \item \textbf{Storage:} worst-case \(O(q n)\) pages (heavy), practical with CoW \(\approx O(n + \sum r)\).
  \item \textbf{Prefetch latency:} depends on PCIe/SSD; model explicitly in expansion cost \(c_i\).
  \item \textbf{Compute:} LLM inference dominates; parallelism multiplies throughput if KV residency is managed.
\end{itemize}

\section{Failure Modes and Mitigations}
\begin{itemize}[nosep]
  \item \textbf{I/O bottlenecks:} mitigate with prefetching and penalizing high \(c_i\) in Eq.~\ref{eq:prob}.
  \item \textbf{Over-branching explosion:} cap branching factor, budgeted search, and prune low-utility nodes.
  \item \textbf{Compression-induced errors:} keep critical nodes high-fidelity and verify before compression.
  \item \textbf{Controller failure modes:} hybrid meta/local controllers and soft consensus for fault tolerance.
\end{itemize}

\section{Limitations and Ethical Considerations}
KVTG increases system complexity, and implementing it requires careful systems engineering. Compression may lead to subtle errors; provenance should be tracked to allow human audit. As with other LLM systems, misuse risks exist (e.g., automating harmful reasoning); we recommend standard safety checks and human-in-the-loop verification for critical domains.

\section{Conclusion}
KV-Cache Thought Graphs offer a path toward non-linear, parallel, and auditable LLM reasoning. By combining immutable node snapshots (with CoW deltas), hybrid distributed controllers, probabilistic expansion policies, and diverse merge strategies, KVTG seeks to unlock parallel reasoning while maintaining verifiability. Our prototypes and experiments aim to quantify practical trade-offs and guide implementation choices for realistic hardware. We invite the community to implement and test KVTG primitives and share empirical findings.

\section*{Acknowledgments}
(Placeholder.)

\appendix
\section{API Sketch and Data Types}
\begin{itemize}[nosep]
  \item \texttt{Node} object: fields \texttt{text, summary, emb, kv\_ptr, meta}.
  \item \texttt{Worker} interface: \texttt{expand(node)}, \texttt{propose()}, \texttt{prefetch(ptr)}.
  \item \texttt{MetaController} API: \texttt{aggregate(proposals)}, \texttt{assign(tasks)}, \texttt{garbage\_collect()}.
\end{itemize}

\section{Pseudocode: End-to-End Loop}
\begin{verbatim}
while not finished:
    # Meta: collect proposals
    proposals = gather_from_workers()
    accepted = meta_controller.aggregate_and_select(proposals)
    # Assign tasks and prefetch
    for a in accepted:
        assign_to_worker(a.worker, a.node)
        worker.prefetch(a.node.kv_ptr)
    # Workers expand asynchronously and report back
\end{verbatim}

\section{Example: Small math-proof experiment plan}
\begin{itemize}[nosep]
  \item Task: short multi-step olympiad-style problems decomposed into lemmas.
  \item Baselines: chain-of-thought (CoT), tree-of-thoughts (ToT).
  \item KVTG config: r=8 (KV delta pages), active VRAM = 4 GB, budget 8 expansions.
  \item Metrics: solution correctness, expansions to solve, wall-clock time.
\end{itemize}

\bibliographystyle{plain}
\begin{thebibliography}{9}
\bibitem{cot}
  (Chain-of-Thought references placeholder.)
\bibitem{vllm}
  (Paged KV and vLLM references placeholder.)
\end{thebibliography}

\end{document}
